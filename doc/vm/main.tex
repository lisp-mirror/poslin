\documentclass[a4paper]{article}

\newcommand{\opdef}[1]{\subsubsection{\texttt{#1}}}

\begin{document}
\title{Poslin VM Documentation}
\author{Thomas Bartscher}
\date{2020-03-10}
\maketitle

\tableofcontents


\section{Type System}
In the VM every object has exactly one type. The types are fixed, new types
cannot be defined.

\subsection{Scalar Types}
\subsubsection{Nothing}
This type is inhabited by one value called ``nothing''.
\subsubsection{Type}
This type is inhabited by one object for each primitive type the VM recognizes.
\subsubsection{Symbol}
Symbols have names and represent an identity. There exist interned symbols and
uninterned symbols. Interned symbols can be reverenced via their name,
uninterned symbols cannot.
\subsubsection{Boolean}
The two boolean values are true and false.
\subsubsection{Comparison}
The comparison values are ``less'', ``greater'', ``equal'' and ``unequal''.
\subsubsection{Precise}
This type is inhabited by rational numbers. Poslin does not differentiate
integers.
\subsubsection{Imprecise}
This type is inhabited by double floats.
\subsubsection{Character}
This type is inhabited by unicode values.
\subsubsection{Empty Stack}
This type is inhabited by the empty stack.
\subsubsection{Elementary Thread}
This type is inhabited by all elementary threads.
\subsubsection{ConstantThread}
This type is inhabited by threads that always return a constant value.

\subsection{Compound Types}
\subsubsection{Binding}
\subsubsection{Stack}
\subsubsection{Set}
\subsubsection{Dictionary}
\subsubsection{Thread}
\subsubsection{HandledThread}
\subsubsection{Exception}


\section{Internal State}
\subsection{The Path}
The path binding is a constant binding containing a stack. That stack is called
the path.
\subsubsection{The Current Environment}
The top of the path must always contain a binding containing a dictionary. That
dictionary is called the current environment, the binding the current
environment binding. The current environment must always have entries for the
symbols \texttt{STACK}, \texttt{OP} and \texttt{IMM}.
\subsubsection{The Current Stack}
The \texttt{STACK} entry in the current environment must map to a binding
containing a stack. This stack is called the current stack, the binding the
current stack binding.
\subsubsection{Current Operations}
The \texttt{OP} entry in the current environment must map to a binding
containing a dictionary. The dictionary is called the current operation
environment, the binding the current operation environment binding.

The current operation environment should map symbols to bindings containing
threads. The symbols mapped by the current operation environment are called
operations.
\subsubsection{Current Immediateness}
The \texttt{IMM} entry in the current environment must map to a binding
containing a set. The set is called the current immediateness environment, the
binding the current immediateness environment binding.

The set should contain symbols who are operations.

\subsection{The Return Stack}
The return stack binding is a constant binding containing a stack. That stack is
called the return stack.

\subsection{The Instruction Pointer}
The instruction pointer points towards the next thread to evaluate or to
nothing. If it points to nothing, execution stops. If it points to a elementary
or constant thread it executes that thread, then checks whether the return stack
is empty. If it is not, the topmost value is popped off the return stack and the
instruction pointer pointed to that value. If the return stack is empty, the vm
stops.


\section{Operations}
\subsection{Control}
\opdef{!}
\opdef{\&}
\opdef{\#}
\opdef{->elem-thread}
\opdef{?}
\opdef{rstack-binding}
\opdef{r<-}
\opdef{r->}

\subsection{Thread}
\opdef{thread-front}
\opdef{thread-back}
\opdef{thread-concat}

\subsection{Path}
\opdef{path-binding}

\subsection{Nothing}
\opdef{.nothing}

\subsection{Boolean}
\opdef{.true}
\opdef{.false}
\opdef{and}
\opdef{or}
\opdef{not}
\opdef{same?}

\subsection{Comparison}
\opdef{.less}
\opdef{.greater}
\opdef{.equal}
\opdef{.unequal}
\opdef{compare}

\subsection{Type}
\opdef{type}

\subsection{Symbol}
\opdef{unique-symbol}
\opdef{symbol-concat}

\subsection{Arithmetic}
\opdef{+}
\opdef{*}
\opdef{negation}
\opdef{reciprocal}
\opdef{log}
\opdef{pow}
\opdef{round}
\opdef{floor}
\opdef{ceiling}
\opdef{->imprecise}

\subsection{Binding}
\opdef{new-binding}
\opdef{retrieve}
\opdef{store}

\subsection{Character}
\opdef{int->char}
\opdef{char->int}

\subsection{Stack}
\opdef{.empty-stack}
\opdef{top}
\opdef{push}
\opdef{drop}
\opdef{swap}

\subsection{Array}
\opdef{new-array}
\opdef{array-lookup}
\opdef{array-set}
\opdef{array-size}
\opdef{array-concat}

\subsection{String}
\opdef{->string}
\opdef{string-lookup}
\opdef{string-set}
\opdef{string-size}
\opdef{string-concat}

\subsection{Set}
\opdef{.empty-set}
\opdef{set-lookup}
\opdef{set-insert}
\opdef{set-drop}
\opdef{set-arbitrary}

\subsection{Dictionary}
\opdef{.empty-dict}
\opdef{dict-lookup}
\opdef{dict-insert}
\opdef{dict-drop}
\opdef{dict-domain}

\subsection{Stream}
\opdef{.stdin}
\opdef{.stdout}
\opdef{open}
\opdef{close}
\opdef{read-char}
\opdef{write-char}

\subsection{Exception}
\opdef{new-exception}
\opdef{throw}
\opdef{handle}
\opdef{exception-message}
\opdef{exception-data}
\opdef{exception-stack}
\opdef{thread-handle}

\subsection{Loading}
\opdef{load}

\end{document}
